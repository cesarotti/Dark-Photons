\section{example\_\-My\-Pythia\-Only\-To\-Hep\-MC.cc}


\begin{DocInclude}\begin{verbatim}1 
2 // Matt.Dobbs@Cern.CH, December 1999
3 // November 2000, updated to use Pythia 6.1
4 // example of generating events with Pythia 
5 // using HepMC/PythiaWrapper.h 
6 // Events are read into the HepMC event record from the FORTRAN HEPEVT 
7 // common block using the IO_HEPEVT strategy -- nothing is done with them.
8 // This program is just used to find the total time required to transfer
9 // from HEPEVT into the HepMC event record.
11 // To Compile: go to the HepMC directory and type:
12 // gmake examples/example_MyPythiaOnlyTo HepMC.exe
13 //
14 // See comments in examples/example_MyPythia.cxx regarding the HEPEVT wrapper.
15 //
16 
17 #include <iostream>
18 #include "HepMC/PythiaWrapper.h"
19 #include "HepMC/IO_HEPEVT.h"
20 #include "HepMC/GenEvent.h"
21 #include "PythiaHelper.h"
22 
23 int main() {    
24     //
25     //........................................HEPEVT
26     // Pythia 6.1 uses HEPEVT with 4000 entries and 8-byte floating point
27     //  numbers. We need to explicitly pass this information to the 
28     //  HEPEVT_Wrapper.
29     //
30     HepMC::HEPEVT_Wrapper::set_max_number_entries(4000);
31     HepMC::HEPEVT_Wrapper::set_sizeof_real(8);
32     //  
33     //........................................PYTHIA INITIALIZATIONS
34     initPythia();
35     //
36     //........................................HepMC INITIALIZATIONS
37     //
38     // Instantiate an IO strategy for reading from HEPEVT.
39     HepMC::IO_HEPEVT hepevtio;
40     //
41     //........................................EVENT LOOP
42     for ( int i = 1; i <= 100; i++ ) {
43         if ( i%50==1 ) std::cout << "Processing Event Number " 
44                                  << i << std::endl;
45         call_pyevnt();      // generate one event with Pythia
46         // pythia pyhepc routine convert common PYJETS in common HEPEVT
47         call_pyhepc( 1 );
48         HepMC::GenEvent* evt = hepevtio.read_next_event();
49         // set number of multi parton interactions
50         evt->set_mpi( pypars.msti[31-1] );
51         //
52         //.......................USER WOULD PROCESS EVENT HERE
53         //
54         // we also need to delete the created event from memory
55         delete evt;
56     }
57     //........................................TERMINATION
58     // write out some information from Pythia to the screen
59     call_pystat( 1 );    
60 
61     return 0;
62 }
63 
64 
65  
\end{verbatim}
\end{DocInclude}
 