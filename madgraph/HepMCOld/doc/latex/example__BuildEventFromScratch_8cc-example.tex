\section{example\_\-Build\-Event\-From\-Scratch.cc}


\begin{DocInclude}\begin{verbatim}1 
2 // Matt.Dobbs@Cern.CH, Feb 2000
3 // Example of building an event and a particle data table from scratch
4 // This is meant to be of use for persons implementing HepMC inside a MC 
5 // event generator
7 // To Compile: go to the HepMC directory and type:
8 // gmake examples/example_BuildEventFromScratch.exe
9 //
10 
11 #include <iostream>
12 
13 #include "VectorConversion.h"
14 #include "HepMC/GenEvent.h"
15 #include "HepMC/ParticleDataTable.h"
16 #include "CLHEP/Vector/LorentzVector.h"
17 
18 // in this example we use the HepMC namespace, so that we do not have to 
19 // precede all HepMC classes with HepMC::
20 
21 // This example also shows how to use the CLHEP Lorentz vector with HepMC2
22 
23 using namespace HepMC;
24 using namespace CLHEP;
25 
26 int main() {
27     //
28     // In this example we will place the following event into HepMC "by hand"
29     //
30     //     name status pdg_id  parent Px       Py    Pz       Energy      Mass
31     //  1  !p+!    3   2212    0,0    0.000    0.000 7000.000 7000.000    0.938
32     //  2  !p+!    3   2212    0,0    0.000    0.000-7000.000 7000.000    0.938
33     //=========================================================================
34     //  3  !d!     3      1    1,1    0.750   -1.569   32.191   32.238    0.000
35     //  4  !u~!    3     -2    2,2   -3.047  -19.000  -54.629   57.920    0.000
36     //  5  !W-!    3    -24    1,2    1.517   -20.68  -20.605   85.925   80.799
37     //  6  !gamma! 1     22    1,2   -3.813    0.113   -1.833    4.233    0.000
38     //  7  !d!     1      1    5,5   -2.445   28.816    6.082   29.552    0.010
39     //  8  !u~!    1     -2    5,5    3.962  -49.498  -26.687   56.373    0.006
40 
41     // first we construct a ParticleDataTable with all the particles we need
42     ParticleDataTable pdt("my particle data table");
43     // create a particle data entry for the proton and add it to pdt at the
44     // same time
45     pdt.insert( new ParticleData( "p+", 2212,   +1, 0.938,  -1, .5 ) );
46     pdt.insert( new ParticleData( "d",  1,  -2./3., 0,      -1, .5 ) );
47     pdt.insert( new ParticleData( "u~", -2, -1./3., 0,      -1, .5 ) );
48     pdt.insert( new ParticleData( "W-", -24,    -1, 80.396,
49                                   clifetime_from_width(2.06), 1 )    );
50     pdt.insert( new ParticleData( "gamma", 22,   0, 0,      -1, 1  ) );
51 
52     // print out the GenParticle Data to the screen
53     pdt.print();
54 
55     // now we build the graph, which will look like
56     //                       p7
57     // p1                   /
58     //   \v1__p3      p5---v4
59     //         \_v3_/       \ 
60     //         /    \        p8
61     //    v2__p4     \ 
62     //   /            p6
63     // p2
64     //
65 
66     // First create the event container, with Signal Process 20, event number 1
67     //
68     // Note that the HepLorentzVectors will be automatically converted to 
69     // HepMC::FourVector within GenParticle and GenVertex
70     GenEvent* evt = new GenEvent( 20, 1 );
71     //
72     // create vertex 1 and vertex 2, together with their inparticles
73     GenVertex* v1 = new GenVertex();
74     evt->add_vertex( v1 );
75     v1->add_particle_in( new GenParticle( HepLorentzVector(0,0,7000,7000),
76                                        2212, 3 ) );
77     GenVertex* v2 = new GenVertex();
78     evt->add_vertex( v2 );
79     v2->add_particle_in( new GenParticle( HepLorentzVector(0,0,-7000,7000),
80                                        2212, 3 ) );
81     //
82     // create the outgoing particles of v1 and v2
83     GenParticle* p3 = 
84         new GenParticle( HepLorentzVector(.750,-1.569,32.191,32.238), 1, 3 );
85     v1->add_particle_out( p3 );
86     GenParticle* p4 = 
87         new GenParticle( HepLorentzVector(-3.047,-19.,-54.629,57.920), -2, 3 );
88     v2->add_particle_out( p4 );
89     //
90     // create v3
91     GenVertex* v3 = new GenVertex();
92     evt->add_vertex( v3 );
93     v3->add_particle_in( p3 );
94     v3->add_particle_in( p4 );
95     v3->add_particle_out( 
96         new GenParticle( HepLorentzVector(-3.813,0.113,-1.833,4.233 ), 22, 1 )
97         );
98     GenParticle* p5 = 
99         new GenParticle( HepLorentzVector(1.517,-20.68,-20.605,85.925), -24,3);
100     v3->add_particle_out( p5 );
101     //
102     // create v4
103     GenVertex* v4 = new GenVertex(HepLorentzVector(0.12,-0.3,0.05,0.004));
104     evt->add_vertex( v4 );
105     v4->add_particle_in( p5 );
106     v4->add_particle_out( 
107         new GenParticle( HepLorentzVector(-2.445,28.816,6.082,29.552), 1,1 )
108         );
109     v4->add_particle_out( 
110         new GenParticle( HepLorentzVector(3.962,-49.498,-26.687,56.373), -2,1 )
111         );
112     //    
113     // tell the event which vertex is the signal process vertex
114     evt->set_signal_process_vertex( v3 );
115     // the event is complete, we now print it out to the screen
116     evt->print();
117     
118     // example conversion back to Lorentz vector
119     // add all outgoing momenta
120     std::cout << std::endl;
121     std::cout << " Add output momenta " << std::endl;
122     HepLorentzVector sum;
123     for ( GenEvent::particle_const_iterator p = evt->particles_begin(); 
124               p != evt->particles_end(); ++p ){
125         if( (*p)->status() == 1 ) {
126             sum += SVtoLV( (*p)->momentum() );
127             (*p)->print();
128         }
129     }
130     std::cout << "Vector Sum: " << sum << std::endl;
131 
132     // now clean-up by deleteing all objects from memory
133     //
134     // deleting the event deletes all contained vertices, and all particles
135     // contained in those vertices
136     delete evt;
137 
138     // delete all particle data objects in the particle data table pdt
139     pdt.delete_all();
140 
141     return 0;
142 }
\end{verbatim}
\end{DocInclude}
 