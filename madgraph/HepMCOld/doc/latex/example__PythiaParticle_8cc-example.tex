\section{example\_\-Pythia\-Particle.cc}


\begin{DocInclude}\begin{verbatim}1 
2 // garren@fnal.gov, July 2006
3 // example of generating events with Pythia
4 // using HepMC/PythiaWrapper.h 
5 // Events are read into the HepMC event record from the FORTRAN HEPEVT 
6 // common block using the IO_HEPEVT strategy and then output to file in
7 // ascii format using the IO_AsciiParticles strategy.
8 //
9 // This is identical to example_MyPythia.cc except that it uses IO_AsciiParticles.
11 // To Compile: go to the examples directory and type:
12 // gmake example_PythiaParticle.exe
13 //
14 // In this example the precision and number of entries for the HEPEVT 
15 // fortran common block are explicitly defined to correspond to those 
16 // used in the Pythia version of the HEPEVT common block. 
17 //
18 // If you get funny output from HEPEVT in your own code, probably you have
19 // set these values incorrectly!
20 //
21 
22 #include <iostream>
23 #include "HepMC/PythiaWrapper.h"
24 #include "HepMC/IO_HEPEVT.h"
25 #include "HepMC/IO_AsciiParticles.h"
26 #include "HepMC/GenEvent.h"
27 #include "PythiaHelper.h"
28     
29 int main() { 
30     //
31     //........................................HEPEVT
32     // Pythia 6.1 uses HEPEVT with 4000 entries and 8-byte floating point
33     //  numbers. We need to explicitly pass this information to the 
34     //  HEPEVT_Wrapper.
35     //
36     HepMC::HEPEVT_Wrapper::set_max_number_entries(4000);
37     HepMC::HEPEVT_Wrapper::set_sizeof_real(8);
38     //
39     //........................................PYTHIA INITIALIZATIONS
40     initPythia();
41 
42     //........................................HepMC INITIALIZATIONS
43     //
44     // Instantiate an IO strategy for reading from HEPEVT.
45     HepMC::IO_HEPEVT hepevtio;
46     //
47     { // begin scope of ascii_io
48         // Instantiate an IO strategy to write the data to file 
49         HepMC::IO_AsciiParticles ascii_io("example_PythiaParticle.dat",std::ios::out);
50         //
51         //........................................EVENT LOOP
52         for ( int i = 1; i <= 100; i++ ) {
53             if ( i%50==1 ) std::cout << "Processing Event Number " 
54                                      << i << std::endl;
55             call_pyevnt();      // generate one event with Pythia
56             // pythia pyhepc routine converts common PYJETS in common HEPEVT
57             call_pyhepc( 1 );
58             HepMC::GenEvent* evt = hepevtio.read_next_event();
59             // add some information to the event
60             evt->set_event_number(i);
61             evt->set_signal_process_id(20);
62             // write the event out to the ascii file
63             ascii_io << evt;
64             // we also need to delete the created event from memory
65             delete evt;
66         }
67         //........................................TERMINATION
68         // write out some information from Pythia to the screen
69         call_pystat( 1 );    
70     } // end scope of ascii_io
71 
72     return 0;
73 }
74 
75 
76  
\end{verbatim}
\end{DocInclude}
 