\section{example\_\-My\-Pythia\-Read.cc File Reference}
\label{example__MyPythiaRead_8cc}\index{example_MyPythiaRead.cc@{example\_\-MyPythiaRead.cc}}
{\tt \#include $<$iostream$>$}\par
{\tt \#include \char`\"{}Hep\-MC/Pythia\-Wrapper.h\char`\"{}}\par
{\tt \#include \char`\"{}Hep\-MC/IO\_\-HEPEVT.h\char`\"{}}\par
{\tt \#include \char`\"{}Hep\-MC/IO\_\-Ascii.h\char`\"{}}\par
{\tt \#include \char`\"{}Hep\-MC/Gen\-Event.h\char`\"{}}\par
{\tt \#include \char`\"{}Pythia\-Helper.h\char`\"{}}\par
\subsection*{Functions}
\begin{CompactItemize}
\item 
int {\bf main} ()
\end{CompactItemize}


\subsection{Function Documentation}
\index{example_MyPythiaRead.cc@{example\_\-My\-Pythia\-Read.cc}!main@{main}}
\index{main@{main}!example_MyPythiaRead.cc@{example\_\-My\-Pythia\-Read.cc}}
\subsubsection{\setlength{\rightskip}{0pt plus 5cm}int main ()}\label{example__MyPythiaRead_8cc_e66f6b31b5ad750f1fe042a706a4e3d4}




Definition at line 28 of file example\_\-My\-Pythia\-Read.cc.

References Hep\-MC::Gen\-Event::event\_\-number(), init\-Pythia(), Hep\-MC::IO\_\-Base\-Class::read\_\-next\_\-event(), Hep\-MC::Gen\-Event::set\_\-event\_\-number(), Hep\-MC::HEPEVT\_\-Wrapper::set\_\-max\_\-number\_\-entries(), Hep\-MC::Gen\-Event::set\_\-signal\_\-process\_\-id(), and Hep\-MC::HEPEVT\_\-Wrapper::set\_\-sizeof\_\-real().