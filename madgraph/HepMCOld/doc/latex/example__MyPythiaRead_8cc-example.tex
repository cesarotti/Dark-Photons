\section{example\_\-My\-Pythia\-Read.cc}


\begin{DocInclude}\begin{verbatim}1 
2 // garren@fnal.gov, January 2007
3 // This example is an extension of example_MyPythia.cc
4 //  
5 // generate events with Pythia, write a file, and read the resulting output
6 // Notice that we use scope to explicitly close the ouput files.
7 // The two output files should be the same size, but because particles are
8 // saved as sets within a vertex, they will be written in arbitrary order.
10 // To Compile: go to the HepMC directory and type:
11 // gmake examples/example_MyPythiaRead.exe
12 //
13 // In this example the precision and number of entries for the HEPEVT 
14 // fortran common block are explicitly defined to correspond to those 
15 // used in the Pythia version of the HEPEVT common block. 
16 //
17 // If you get funny output from HEPEVT in your own code, probably you have
18 // set these values incorrectly!
19 //
20 
21 #include <iostream>
22 #include "HepMC/PythiaWrapper.h"
23 #include "HepMC/IO_HEPEVT.h"
24 #include "HepMC/IO_Ascii.h"
25 #include "HepMC/GenEvent.h"
26 #include "PythiaHelper.h"
27 
28 int main() { 
29     //
30     //........................................HEPEVT
31     // Pythia 6.3 uses HEPEVT with 4000 entries and 8-byte floating point
32     //  numbers. We need to explicitly pass this information to the 
33     //  HEPEVT_Wrapper.
34     //
35     HepMC::HEPEVT_Wrapper::set_max_number_entries(4000);
36     HepMC::HEPEVT_Wrapper::set_sizeof_real(8);
37     //
38     //........................................PYTHIA INITIALIZATIONS
39     initPythia();
40 
41     //........................................HepMC INITIALIZATIONS
42     //
43     // Instantiate an IO strategy for reading from HEPEVT.
44     HepMC::IO_HEPEVT hepevtio;
45     //
46     //........................................define the output scope
47     {
48         // Instantial an IO strategy to write the data to file - it uses the 
49         //  same ParticleDataTable
50         HepMC::IO_Ascii ascii_io("example_MyPythiaRead.dat",std::ios::out);
51         //
52         //........................................EVENT LOOP
53         for ( int i = 1; i <= 100; i++ ) {
54             if ( i%50==1 ) std::cout << "Processing Event Number " 
55                                      << i << std::endl;
56             call_pyevnt();      // generate one event with Pythia
57             // pythia pyhepc routine converts common PYJETS in common HEPEVT
58             call_pyhepc( 1 );
59             HepMC::GenEvent* evt = hepevtio.read_next_event();
60             // add some information to the event
61             evt->set_event_number(i);
62             evt->set_signal_process_id(20);
63             // write the event out to the ascii file
64             ascii_io << evt;
65             // we also need to delete the created event from memory
66             delete evt;
67         }
68         //........................................TERMINATION
69         // write out some information from Pythia to the screen
70         call_pystat( 1 );    
71     }  // ascii_io destructor is called here
72     //
73     //........................................define an input scope
74     {
75         // now read the file we wrote
76         HepMC::IO_Ascii ascii_in("example_MyPythiaRead.dat",std::ios::in);
77         HepMC::IO_Ascii ascii_io2("example_MyPythiaRead2.dat",std::ios::out);
78         int icount=0;
79         HepMC::GenEvent* evt = ascii_in.read_next_event();
80         while ( evt ) {
81             icount++;
82             if ( icount%50==1 ) std::cout << "Processing Event Number " << icount
83                                           << " its # " << evt->event_number() 
84                                           << std::endl;
85             // write the event out to the ascii file
86             ascii_io2 << evt;
87             delete evt;
88             ascii_in >> evt;
89         }
90         //........................................PRINT RESULT
91         std::cout << icount << " events found. Finished." << std::endl;
92     } // ascii_io2 and ascii_in destructors are called here
93 
94     return 0;
95 }
96 
97 
98  
\end{verbatim}
\end{DocInclude}
 