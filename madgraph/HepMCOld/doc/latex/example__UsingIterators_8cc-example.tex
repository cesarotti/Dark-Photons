\section{example\_\-Using\-Iterators.cc}


\begin{DocInclude}\begin{verbatim}1 
2 // Matt.Dobbs@Cern.CH, Feb 2000
3 // This example shows low to use the particle and vertex iterators
5 // To Compile: go to the HepMC directory and type:
6 // gmake examples/example_UsingIterators.exe
7 //
8 
9 #include "HepMC/IO_Ascii.h"
10 #include "HepMC/GenEvent.h"
11 #include <math.h>
12 #include <algorithm>
13 #include <list>
14 
16 
20 class IsPhoton {
21 public:
23     bool operator()( const HepMC::GenParticle* p ) { 
24         if ( p->pdg_id() == 22 
25              && p->momentum().perp() > 10. ) return 1;
26         return 0;
27     }
28 };
29 
31 
34 class IsW_Boson {
35 public:
37     bool operator()( const HepMC::GenParticle* p ) { 
38         if ( abs(p->pdg_id()) == 24 ) return 1;
39         return 0;
40     }
41 };
42 
44 
47 class IsFinalState {
48 public:
50     bool operator()( const HepMC::GenParticle* p ) { 
51         if ( !p->end_vertex() && p->status()==1 ) return 1;
52         return 0;
53     }
54 };
55 
56 int main() {
57     { // begin scope of ascii_in
58         // an event has been prepared in advance for this example, read it
59         // into memory using the IO_Ascii input strategy
60         HepMC::IO_Ascii ascii_in("example_UsingIterators.txt",std::ios::in);
61         if ( ascii_in.rdstate() == std::ios::failbit ) {
62             std::cerr << "ERROR input file example_UsingIterators.txt is needed "
63                       << "and does not exist. "
64                       << "\n Look for it in HepMC/examples, Exit." << std::endl;
65             return 1;
66         }
67 
68         HepMC::GenEvent* evt = ascii_in.read_next_event();
69 
70         // if you wish to have a look at the event, then use evt->print();
71 
72         // use GenEvent::vertex_iterator to fill a list of all 
73         // vertices in the event
74         std::list<HepMC::GenVertex*> allvertices;
75         for ( HepMC::GenEvent::vertex_iterator v = evt->vertices_begin();
76               v != evt->vertices_end(); ++v ) {
77             allvertices.push_back(*v);
78         }
79 
80         // we could do the same thing with the STL algorithm copy
81         std::list<HepMC::GenVertex*> allvertices2;
82         copy( evt->vertices_begin(), evt->vertices_end(), 
83               back_inserter(allvertices2) );
84 
85         // fill a list of all final state particles in the event, by requiring
86         // that each particle satisfyies the IsFinalState predicate
87         IsFinalState isfinal;
88         std::list<HepMC::GenParticle*> finalstateparticles;
89         for ( HepMC::GenEvent::particle_iterator p = evt->particles_begin();
90               p != evt->particles_end(); ++p ) {
91             if ( isfinal(*p) ) finalstateparticles.push_back(*p);
92         }
93 
94         // an STL-like algorithm called HepMC::copy_if is provided in the
95         // GenEvent.h header to do this sort of operation more easily,
96         // you could get the identical results as above by using:
97         std::list<HepMC::GenParticle*> finalstateparticles2;
98         HepMC::copy_if( evt->particles_begin(), evt->particles_end(), 
99                         back_inserter(finalstateparticles2), IsFinalState() );
100 
101         // lets print all photons in the event that satisfy the IsPhoton criteria
102         IsPhoton isphoton;
103         for ( HepMC::GenEvent::particle_iterator p = evt->particles_begin();
104               p != evt->particles_end(); ++p ) {
105             if ( isphoton(*p) ) (*p)->print();
106         }
107 
108         // the GenVertex::particle_iterator and GenVertex::vertex_iterator
109         // are slightly different from the GenEvent:: versions, in that
110         // the iterator starts at the given vertex, and walks through the attached 
111         // vertex returning particles/vertices.
112         // Thus only particles/vertices which are in the same graph as the given
113         // vertex will be returned. A range is specified with these iterators,
114         // the choices are:
115         //    parents, children, family, ancestors, descendants, relatives 
116         // here are some examples.
117 
118         // use GenEvent::particle_iterator to find all W's in the event,
119         // then 
120         // (1) for each W user the GenVertex::particle_iterator with a range of
121         //     parents to return and print the immediate mothers of these W's.
122         // (2) for each W user the GenVertex::particle_iterator with a range of
123         //     descendants to return and print all descendants of these W's.
124         IsW_Boson isw;
125         for ( HepMC::GenEvent::particle_iterator p = evt->particles_begin();
126               p != evt->particles_end(); ++p ) {
127             if ( isw(*p) ) {
128                 std::cout << "A W boson has been found: " << std::endl;
129                 (*p)->print();
130                 // return all parents
131                 // we do this by pointing to the production vertex of the W 
132                 // particle and asking for all particle parents of that vertex
133                 std::cout << "\t Its parents are: " << std::endl;
134                 if ( (*p)->production_vertex() ) {
135                     for ( HepMC::GenVertex::particle_iterator mother 
136                               = (*p)->production_vertex()->
137                               particles_begin(HepMC::parents);
138                           mother != (*p)->production_vertex()->
139                               particles_end(HepMC::parents); 
140                           ++mother ) {
141                         std::cout << "\t";
142                         (*mother)->print();
143                     }
144                 }
145                 // return all descendants
146                 // we do this by pointing to the end vertex of the W 
147                 // particle and asking for all particle descendants of that vertex
148                 std::cout << "\t\t Its descendants are: " << std::endl;
149                 if ( (*p)->end_vertex() ) {
150                     for ( HepMC::GenVertex::particle_iterator des 
151                               =(*p)->end_vertex()->
152                               particles_begin(HepMC::descendants);
153                           des != (*p)->end_vertex()->
154                               particles_end(HepMC::descendants);
155                           ++des ) {
156                         std::cout << "\t\t";
157                         (*des)->print();
158                     }
159                 }
160             }
161         }
162         // cleanup
163         delete evt;
164         // in analogy to the above, similar use can be made of the
165         // HepMC::GenVertex::vertex_iterator, which also accepts a range.
166     } // end scope of ascii_in
167 
168     return 0;
169 }
\end{verbatim}
\end{DocInclude}
 